\documentclass[a4paper,12pt]{article}

\def\today{\number\day.~\ifcase\month\or
    januar \or februar \or mart \or april \or maj \or jun \or jul \or
    avgust \or septembar \or oktobar \or novembar \or decembar \fi \space
    \number\year.}

\def\figurename{Slika}

\usepackage{indentfirst}
\addtolength{\oddsidemargin}{-15mm}
\addtolength{\textwidth}{25mm}
\addtolength{\textheight}{35mm}
\setlength{\topmargin}{-0.5in}
\pdfpagewidth=\paperwidth
\pdfpageheight=\paperheight

\def\d{d\kern-0.4em\symbol{22}\kern-0.1em}
\def\D{D\kern-0.8em\raise0.2ex\hbox{--}\kern0.3em}

\author{Karaka\v s Zlatko $<$\href{mailto:zlatko.karakas@gmail.com}{zlatko.karakas@gmail.com}$>$}
\title{{\bf Uputstvo za SWOS Picture Editor}\\v0.99.2023}

\usepackage[pdftex,colorlinks,bookmarks=false,pdfstartview=FitH]{hyperref}
\usepackage{graphicx}
\usepackage{float}
\usepackage{enumitem}
\usepackage{longtable}
\usepackage[T1]{fontenc}

\setcounter{secnumdepth}{5}

\begin{document}

\hyphenation{do-zvo-lje-ne}
\hyphenation{skra-\'ce-no}
\hyphenation{naj-vid-lji-vi-ja}

\maketitle

\section{Uvod}
Hvala \v sto koristite {\bf SWOS Ure\dj iva\v c Slika}, odnosno {\bf SWOS Picture Editor}, skra\'ceno {\bf SWPE}.
Ovaj program je napisan da omogu\'ci izmenu i zamenu grafike u igri. To \v cini menjaju\'ci originalne datoteke igre.

Prva verzija programa nastala je davne 2002., i na neki na\v cin predstavlja deo {\bf SWOS} istorije. Napisana
je prvenstveno kao alatka za istra\v zivanje i eksperimentisanje, naro\v cito fajlova sa podacima, i
sadr\v zi u sebi mnoga {\bf SWOS} otkri\'ca do kojih sam do\v sao godinama. Tokom vremena aplikacija je
nadogra\dj ivana stihijski, bez jasnog pravca, kojekakvim zakrpama pa \v cak i delovima koda samog {\bf SWOS}-a,
te ne \v cudi \v sto je grafi\v cki interfejs trpeo: uglavnom je zasnovan na kori\v s\'cenju tastature, iako ova
verzija uvodi upotrebu mi\v sa za izmenu terena.

Ovo uputstvo navodi sve komande i prikazuje razne korisne na\v cine za upotrebu aplikacije. Kao dodatak, postoji i
\href{https://youtu.be/MOIL5FJbcWQ}{kratak video demo} koji prikazuje komande za ure\dj ivanje terena.

Budite pa\v zljivi da ne pokre\'cete {\bf SWOS} dok koristite {\bf SWPE}, jer zaklju\v cava grafi\v cke fajlove
i spre\v cava {\bf SWOS} da im pristupi.

\newpage
\section{Re\v zimi rada programa}

Radi bolje organizacije, program je podeljen u pet delova:
{\bf SPRITE}, {\bf PICTURE}, {\bf PITCH}, {\bf PATTERN} i {\bf REPLAY}
({\bf SPRAJTOVI}, {\bf SLIKE}, {\bf TERENI}, {\bf PLO\v CICE} i {\bf REPRIZE}).
Svaki deo ima druga\v ciju namenu i sopstvene komande. Pritisnite
{\bf F1} u bilo kom modu da vidite koji se tasteri koriste.

%\vspace{baselineskip}

\noindent Komande dostupne na svim ekranima su:

%\vspace{baselineskip}

\begin{tabular}{lll}
{\bf F1}          &--& uklju\v ci/isklju\v ci pomo\'c\\
{\bf F5}          &--& re\v zim sprajtova\\
{\bf F6}          &--& re\v zim slika\\
{\bf F7}          &--& re\v zim terena\\
{\bf F8}          &--& re\v zim plo\v cica\\
{\bf F9}          &--& re\v zim repriza\\
{\bf ALT + ENTER} &--& zauzmi ceo ekran/radi u prozoru\footnotemark[1]\\
{\bf ESCAPE}      &--& izlazak iz programa\\
{\bf A}           &--& informacije o programu
\end{tabular}

\footnotetext[1]{Prikaz preko \v citavog ekrana ne radi, i popravka nije u planu}

\begin{verbatim}
\end{verbatim}

Glavni prozor se mo\v ze pomerati pomeranjem mi\v sa dok se dr\v zi levo dugme (osim ukoliko se ne nalazimo
u re\v zimu izmene terena). Prozor se mo\v ze smanjiti duplim levim klikom dok je pritisnut taster {\bf CONTROL}.

\newpage
\subsection{Re\v zim sprajtova}

Re\v zim sprajtova je bio sve \v sto je ovaj program mogao da radi u po\v cetku. Tada se zvao {\bf SWSPR} --
{\bf SWOS Sprite Viewer}. Sada je to samo jedan od podr\v zanih re\v zima rada i slu\v zi za pregled i zamenu
sprajtova. Postoji 1334 sprajtova u {\bf SWOS}-u. Indeksni fajl je SPRITE.DAT, a fajlovi sa podacima su CHARSET.DAT,
SCORE.DAT, TEAM1.DAT, TEAM2.DAT, TEAM3.DAT, GOAL1.DAT, BENCH.DAT.\\

\vspace{\baselineskip}

\noindent Komande:\\

\vspace{\baselineskip}

\begin{tabular}{lll}
{\bf SPACE/BACKSPACE}\\
{\bf STRELICE DESNO/LEVO} &--& naredni/prethodni sprajt\\
{\bf G}                   &--& direktan skok na sprajt\\
{\bf PAGE UP/DOWN}        &--& 10 sprajtova napred/nazad\\
{\bf 0..9}                &--& promena boje pozadine\\
{\bf HOME/END}            &--& prvi/poslednji sprajt\\
{\bf +/-}                 &--& uve\'canje/umanjenje slike\\
{\bf F2}                  &--& snimi trenutni sprajt u bitmapu\\
{\bf SHIFT + F2}          &--& snimi sve sprajtove\\
{\bf I}                   &--& uklju\v ci/isklju\v ci informacije o sprajtu\\
{\bf INSERT}              &--& zameni trenutni sprajt sa\\
                          &  & odgovaraju\'com bitmapom\\
{\bf SHIFT + INSERT}      &--& zameni sve sprajtove odgovaraju\'cim\\
                          &  & bitmapama iz trenutnog direktorijuma\\
{\bf CTRL + R}            &--& povratak na snimljeno stanje\\
{\bf CTRL + T}            &--& prebacivanje izme\dj u team2.dat i team3.dat
\end{tabular}

\begin{verbatim}

\end{verbatim}

Strelice su tu za prolazak kroz sprajtove. Za brzo pozicioniranje na odre\dj eni
sprajt pritisnite {\bf G} i otkucajte njegov broj.

{\bf F2} snima sprajt na disk u bitmap formatu radi lak\v se izmene. Ukoliko sprajt ima koordinate centra, tekst
fajl sa istim osnovnim imenom \'ce biti kreiran, i sadr\v za\'ce x koordinatu centra u prvoj i y koordinatu centra
u drugoj liniji. Ukoliko je centar na (0, 0) tekst fajl ne\'ce biti kreiran.

Taster {\bf INSERT} \'ce prouzrokovati izmenu sprajta sadr\v zajem spolja\v snjeg fajla. Snimanje i izmena velikog broja
sprajtova obavlja se istim tasterom, ali uz pritisnuto {\bf SHIFT}. {\bf SHIFT + F2} snima sve sprajtove na
disk, dok {\bf SHIFT + INSERT} u\v citava sprajtove iz svih fajlova koje mo\v ze da prona\dj e u trenutnom
direktorijumu.

Pri startu programa sprajtovi 644 -- 946 bi\'ce mapirani na fajl team2.dat. Pritiskom na
{\bf CTRL + T}, sprajtovi 644 -- 946 bi\'ce mapirani na fajl team3.dat, tako da
i on mo\v ze biti izmenjen. Ponovnim pritiskom na tastere vra\'ca se fajl
team2.dat.

\subsubsection{Izmena sprajtova}

Pri zameni sprajta, program tra\v zi bitmapu sa imenom u formi \verb|spr|{\it nnnn}\verb|.bmp|, gde je
{\it nnnn} broj trenutnog sprajta (npr. \verb|spr0007.bmp|). Bitmapa mora biti osmobitna sa paletom od 256 boja,
bez kompresije. Dimenzije moraju biti: 0 $\le$ \v sirina $<$ 320, 0 $\le$ visina $<$ 200.

Preporu\v ceni na\v cin izmene sprajtova je:

\begin{enumerate}
\item Prona\dj ite sprajt koji \v zelite da izmenite (trebalo bi da je prikazan u programu)
\item Zapamtite njegov broj (prikazan u gornjem levom uglu)
\item Pritisnite {\bf F2} -- sprajt je sada snimljen na disk kao bitmapa
\item Prona\dj ite bitmapu, koja je u istom direktorijumu, i po\v cinje sa ``spr'', a zavr\v sava se sa
\v cetiri cifre broja sprajta
\item U\v citajte bitmapu u svoj omiljeni softver za obradu slika i napravite \v zeljene izmene
\item Vratite se u {\bf SWOS Picture Editor}, prona\dj ite polazni sprajt i pritisnite {\bf INSERT} na tastaturi
\item Pritisnite {\bf ESCAPE}, pritisnite {\bf Y} kao odgovor na pitanje da li \v zelite da snimite promene
\item Pokrenite {\bf SWOS} i u\v zivajte u va\v soj novoj grafici\ldots
\end{enumerate}

Budite pa\v zljivi ako menjate veli\v cinu sprajtova -- ovo mo\v ze imati nepredvidive posledice, naro\v cito pri
uve\'canju. Prilikom testiranja sve je uglavnom bilo u redu, ali se povremeno de\v savalo da {\bf SWOS}
nasumi\v cno pukne zbog unesenih izmena.

Pri obradi bitmapa postoji par stvari na koje treba da obratite pa\v znju. Providna boja pozadine \'ce biti boja
broj 17 u paleti. Ukoliko \v zelite da dodate providne povr\v sine sprajtu, jednostavno crtajte ovom bojom.
Mo\v zete \v cak i da ``obri\v sete'' sprajt tako \v sto \'cete ga kompletno obojiti providnom bojom.
Providna boja \'ce biti boja pozadine u trenutku snimanja, tako da je najbolje da izaberete boju koja se ne
pojavljuje u sprajtu.

Tako\dj e treba da znate da \'ce odre\dj ene boje biti konvertovane u zavisnosti od boje ko\v ze igra\v ca i boje
dresova timova koji igraju:

\begin{tabular}{lll}
 0 &--& bez promene\\
 1 &--& bez promene\\
 2 &--& bez promene\\
 3 &--& bez promene\\
 4 &--& boja ko\v ze (svetlija nijansa)\\
 5 &--& boja ko\v ze (srednja nijansa)\\
 6 &--& boja ko\v ze (tamnija nijansa)\\
 7 &--& anulira se\\
 8 &--& bez promene\\
 9 &--& boja kose (srednja nijansa)\\
10 &--& osnovna boja dresa\\
11 &--& boja pruga na dresu (zamenjena sa 10 u slu\v caju vertikalnih pruga)\\
12 &--& boja kose (tamnija nijansa)\\
13 &--& boja kose (svetlija nijansa)\\
14 &--& boja \v sorca\\
15 &--& boja \v carapa
\end{tabular}

\newpage
\subsection{Re\v zim slika}

Osnovni cilj ovog re\v zima je pregled *.256 slika. Mogu\'ce ih je snimati u
bitmape pritiskom na {\bf F2}.\\

\vspace{\baselineskip}

\noindent Komande:\\

\vspace{\baselineskip}

\begin{tabular}{lll}
{\bf SPACE/BACKSPACE}\\
{\bf STRELICE DESNO/LEVO} &--& naredna/prethodna slika\\
{\bf F2}                  &--& snimi sliku kao bitmapu\\
{\bf F3}                  &--& ponovo u\v citaj sliku\\
{\bf +/-}                 &--& osvetli/zatamni sliku\\
\end{tabular}

\begin{figure}[H]
  \begin{center}
    \includegraphics{images/picture-mode}
    \caption{Re\v zim slika}
  \end{center}
\end{figure}

\newpage
\subsection{Re\v zim terena}

Ovaj re\v zim omogu\'cava korisniku pregled \v citavog terena, kao i njegovu
izmenu. Teren je 42 x 53 matrica sastavljena od sli\v cica dimenzija 16 x 16 (nazva\'cemo ih plo\v cice).
Pritiskom na {\bf CONTROL + E}, ulazi se u re\v zim izmena, i pojavljuje se
svetle\'ci kursor. Videti \ref{editingPitch}.

Za prikaz brojeva plo\v cica pritisnite {\bf CONTROL}+{\bf N}. Pritisnite {\bf CONTROL}+{\bf N} ponovo da
isklju\v cite. Pritiskom na {\bf CONTROL}+{\bf U} dobijate prikaz koliko je puta svaka plo\v cica upotrebljena na
terenu. Isklju\v cuje se ponovnim pritiskom.

Pri snimanju i u\v citavanju terena, \v sablon za ime je jednostavno:

\verb|pitch|$<${\it broj terena}$>$\verb|.bmp|, npr. \verb|pitch3.bmp|.

\begin{figure}[H]
  \begin{center}
    \includegraphics[width=320px,height=200px]{images/pitch-mode}
    \caption{Re\v zim terena}
  \end{center}
\end{figure}

\begin{figure}[H]
  \begin{center}
    \includegraphics[width=320px,height=200px]{images/pitch-mode-tile-numbers}
    \caption{Re\v zim terena sa prikazanim brojevima plo\v cica}
  \end{center}
\end{figure}

\pagebreak
\begin{samepage}
\noindent Komande:

\vspace{\baselineskip}

\begin{longtable}{@{\extracolsep{\fill}}lll}
{\bf STRELICE}          &--& pomeranje terena\\
{\bf PAGE UP/DOWN}      &--& pomeranje gore/dole za pribli\v zno\\
                        &  & pola ekrana\\
{\bf SPACE/BACKSPACE}   &--& prika\v zi naredni/prethodni teren\\
{\bf HOME/END}          &--& prika\v zi gornji-levi/donji-desni\\
                        &  & ugao terena\\
{\bf +/-}               &--& promena uslova na terenu\\
                        &  & (smrznut, blatnjav itd.)\\
{\bf CTRL + N}          &--& prika\v zi/ukloni brojeve plo\v cica\\
{\bf CONTROL}+{\bf U}   &--& prika\v zi/ukloni koliko puta\\
                        &  & su plo\v cice upotrebljene\\
{\bf CONTROL}+{\bf Z}   &--& poni\v sti prethodnu akciju\\
{\bf CONTROL}+{\bf Y}   &--& ponovi poslednje poni\v stenu akciju\\
{\bf CONTROL}+{\bf S}   &--& snimi izmene\\
{\bf CONTROL}+{\bf SHIFT}+{\bf S} &--& prisilno snimanje\\
                        &  & (\v cak i ako nema izmena)\\
{\bf CONTROL}+{\bf R}   &--& u\v citaj stanje sa diska\\
{\bf CONTROL}+{\bf O}   &--& optimizuj plo\v cice terena\\
{\bf CTRL + E}          &--& u\dj i/iza\dj i iz re\v zima izmena\\
{\bf F2}                &--& snimi teren u bitmapu\\
{\bf INSERT}            &--& pro\v citaj teren iz odgovaraju\'ce\\
                        &  & bitmape\\
U re\v zimu izmena:     &  &\\
{\bf CTRL + STRELICE GORE/DOLE} &--& promeni broj plo\v cice\\
                        &  & za plus/minus jedan\\
{\bf G}                 &--& unesi broj trenutne plo\v cice sa tastature\\
{\bf C}                 &--& izmeni sve instance trenutne plo\v cice
\end{longtable}
\end{samepage}

\subsubsection{Izmena terena}\label{editingPitch}
Pritisak na {\bf CONTROL+E} prebacuje prikaz terena u re\v zim izmena. Najvidljivija izmena je prikaz
mre\v ze koja ozna\v cava granice izme\dj u plo\v cica, \v sto ih \v cini mnogo uo\v cljivijim.

Klju\v cni koncept u re\v zimu izmena je kursor. On treperi i ozna\v cava plo\v cicu nad kojom se nalazi kao
trenutno izabranu plo\v cicu. Mnoge operacije koriste trenutnu plo\v cicu kao odredi\v ste. Kursor se mo\v ze
pomerati po terenu mi\v sem ili preko strelica na tastaturi.

Treba uvideti da se u ovom re\v zimu samo menja to koje se plo\v cice prikazuju gde na terenu, ali ne i njihov
sadr\v zaj (pikseli). Za to imamo \nameref{tileMode}.

\begin{figure}[H]
  \begin{center}
    \includegraphics[width=1.0\textwidth]{images/pitch-edit-mode}
    \caption{Re\v zim izmena terena gde kursor ozna\v cava trenutnu plo\v cicu }
  \end{center}
\end{figure}

\paragraph{Upotreba mi\v sa}\mbox{}
\vspace{\baselineskip}

Najnovija verzija programa koristi mi\v s za navigaciju i izmenu terena. Dr\v zanje levog dugmeta i pomeranje mi\v sa \'ce
izvr\v siti pomeranje terena. Vrtenje to\v cki\'ca mi\v sa \'ce pomerati teren gore/dole.

Ako kliknemo na plo\v cicu dok dr\v zimo {\bf CONTROL}+{\bf SHIFT}, plo\v cica postaje ozna\v cena tako \v sto
dobija plavi okvir i crta se tamnijim bojama. Svaka plo\v cica sa tim brojem na terenu \'ce tako\dj e biti
ozna\v cena. Ovo mo\v ze biti veoma korisno za razlikovanje sli\v cnih plo\v cica i otkrivanje propu\v stenih.
Ponovo kliknite na ozna\v cenu plo\v cicu dok dr\v zite {\bf CONTROL}+{\bf SHIFT} da je vratite u normalu.

Za privremeni prikaz broja konkretne plo\v cice bez uklju\v civanja sveukupnog prikaza kliknite na nju srednjim dugmetom
mi\v sa. Prikaz je vidljiv samo dok se dr\v zi dugme. Isti efekat se posti\v ze dr\v zanjem tastera {\bf TAB}
na tastaturi.

Klikom na plo\v cicu dok se dr\v zi {\bf ALT} program se prebacuje u re\v zim plo\v cica sa prikazanom trenutnom
plo\v cicom.

Sve ove operacije, osim pomeranja terena mi\v sem, rade i izvan re\v zima izmena.

\paragraph{Izmena plo\v cica}\mbox{}
\vspace{\baselineskip}

Postoji nekoliko na\v cina za izmenu brojeva plo\v cica. U svim slu\v cajevima osim slikanja plo\v cicom prvi
korak je da se kursor postavi na plo\v cicu koja \'ce biti izmenjena.

\begin{itemize}
\setlength\itemsep{0em}
\item \v Setnja kroz plo\v cice: {\bf CONTROL}+{\bf GORE}/{\bf DOLE} ili {\bf CONTROL}+to\v cki\'c mi\v sa
\item Direktno uno\v senje broja plo\v cice: {\bf G} ili dupli klik levim dugmetom mi\v sa
\item Masovna izmena {\bf C}
\item Slikanje plo\v cicom: desni klik mi\v sem ili {\bf SHIFT+STRELICE}
\end{itemize}

{\bf CONTROL}+{\bf GORE}/{\bf DOLE} prolazi kroz plo\v cice na izabranom mestu jednu po jednu. {\bf GORE}
pove\'cava broj plo\v cice a {\bf DOLE} smanjuje. Ovo mo\v ze biti veoma korisno ako su plo\v cice pore\dj ane po
redosledu pojavljivanja. Isti efekat se posti\v ze mi\v sem, kada se dr\v zi {\bf CONTROL} i vrti to\v cki\'c
mi\v sa. Skrolovanje na gore pove\'cava broj plo\v cice, a skrolovanje na dole smanjuje.

Ukoliko \v zelite da podesite ta\v can broj plo\v cice mo\v zete da pritisnete {\bf G} na tastaturi. {\bf SWPE}
\'ce aktivirati unos broja, i ako je uneseni broj ispravan dodeliti ga trenutno izabranom polju. Dupli klik na
plo\v cicu radi istu stvar.

Kada se pritisne {\bf C}, {\bf SWPE} pita za broj plo\v cice, i ako je ispravan, menja svaku plo\v cicu koja ima
isti broj kao trenutna sa unesenom. Na primer, ako je trenutna plo\v cica broj 40, a unesena 41, ova komanda \'ce
promeniti svaku plo\v cicu broj 40 u plo\v cicu broj 41.

\begin{figure}[H]
  \begin{center}
    \includegraphics[width=1.0\textwidth]{images/pitch-mode-mass-replace-before}
    \caption{Ozna\v cena plo\v cica \#40 koja \'ce uskoro biti zamenjena}
  \end{center}
\end{figure}

\begin{figure}[H]
  \begin{center}
    \includegraphics[width=1.0\textwidth]{images/pitch-mode-mass-replace-after}
    \caption{Nakon zamene plo\v cice \#40 plo\v cicom \#41}
  \end{center}
\end{figure}


\paragraph{Slikanje plo\v cicom}\mbox{}
\vspace{\baselineskip}

Uz pomo\'c mi\v sa mo\v zete da ``slikate plo\v cicom'': kliknite desnim dugmetom mi\v sa na plo\v cicu koju
\v zelite da duplirate, i pomerajte mi\v s (bez pu\v stanja dugmeta). Prva plo\v cica na koju je kliknuto \'ce
dobiti crveni okvir, ozna\v cavaju\'ci je kao glavnu ili ``pe\v cat'' plo\v cicu. Svaka plo\v cica preko koje
se pre\dj e mi\v sem postaje ista kao pe\v cat plo\v cica. Dr\v zanje {\bf SHIFT} tastera ``slikanje''
ograni\v cava na prave linije.

Drugi na\v cin da se podesi izvorna plo\v cica je da se na nju klikne levim dugmetom mi\v sa dok se dr\v zi
{\bf CONTROL}. Ovaj na\v cin je koristan kada se \v zeli popuniti podru\v cje koje nije susedno pe\v cat-plo\v cici.

Isto se mo\v ze posti\'ci i putem tastature:
\begin{itemize}[noitemsep,topsep=0pt]
\item {\bf CONTROL}+{\bf ENTER} ozna\v cava ili otkazuje pe\v cat plo\v cicu
\item {\bf SHIFT}+{\bf STRELICE} tasteri pomeraju kursor i dupliraju glavnu plo\v cicu
\end{itemize}

\begin{figure}[H]
  \begin{center}
    \includegraphics[width=1.0\textwidth]{images/pitch-mode-tile-painting}
    \caption{Slikanje plo\v cicom, primetimo pe\v cat plo\v cicu ozna\v cenu crvenim okvirom}
  \end{center}
\end{figure}

\paragraph{Brisanje plo\v cica}\mbox{}
\vspace{\baselineskip}

Izabrana plo\v cica se mo\v ze obrisati pritiskom na {\bf DELETE} taster na tastaturi. Plo\v cica \'ce biti
potpuno obrisana sa terena. U svako mesto gde se pojavljuje bi\'ce upisana nulta plo\v cica.

\paragraph{Optimizacija terena}\mbox{}
\vspace{\baselineskip}

{\bf CONTROL}+{\bf O} aktivira optimizaciju terena. Ovo \'ce dodeliti brojeve plo\v cicama redom onako kako se
pojavljuju na terenu. Plo\v cice koje se ne pojavljuju na terenu bi\'ce obrisane. Animirane plo\v cice se
ignori\v su (te efektivno bri\v su).

Ovaj proces kreira minimalnu predstavu trenutnog terena i obi\v cno dolazi na kraju, kada ste zadovoljni kako
stvari izgledaju.

\paragraph{Snimanje izmena}\mbox{}
\vspace{\baselineskip}

Da snimite izmene terena na disk pritisnite {\bf CONTROL}+{\bf S}. U slu\v caju da nikakve izmene nisu
detektovane ni\v sta se ne\'ce dogoditi. Da biste prisilili program da ipak izvr\v si snimanje bez obzira da li
ima ili nema promena, pritisnite {\bf CONTROL}+{\bf SHIFT}+{\bf S}.

\pagebreak
\paragraph{Poni\v stavanje/ponavljanje poni\v stenih akcija}\mbox{}
\vspace{\baselineskip}

Bilo koja operacija nad terenom mo\v ze biti poni\v stena i zatim ponovljena prakti\v cno neograni\v cen broj
puta. {\bf CONTROL}+{\bf Z} poni\v stava poslednju operaciju, a {\bf CONTROL}+{\bf Y} ponavlja poslednje
poni\v stenu operaciju.

Ovo je skora\v snji dodatak programu i nije bio ozbiljno testiran. Snimajte \v cesto da ne bude posle kuku lele.

\newpage
\subsection{Re\v zim plo\v cica} \label{tileMode}


U ovom re\v zimu se pregledaju i menjaju pojedina\v cne \v plo\v cice. Plo\v cice su iscrtane u sredini prozora, a
njihov broj, broj terena i koliko su puta upotrebljene u gornjem levom uglu. Neupotrebljene plo\v cice su
ozna\v cene sa \verb|**UNUSED**|.

Za izmenu piksela u plo\v cicama mo\v zemo koristiti slede\'ce korake:
\begin{itemize}[noitemsep]
\item pozicionirajte se na \v zeljeni teren i plo\v cicu
\item pritisnite taster {\bf F2} da bi snimili plo\v cicu u bitmapu
\item izmenite bitmapu u grafi\v ckom editoru
\item u\v citajte izmenjenu bitmapu nazad u plo\v cicu putem {\bf INSERT} tastera
\end{itemize}

Ime fajla je u obliku: \verb|pt|$<${\it broj terena}$>$\verb|-|$<${\it broj plo\v cice}$>$\verb|.bmp|,
gde je $<${\it broj terena}$>$ jedna cifra a $<${\it broj plo\v cice}$>$ je predstavljen sa \v cetiri cifre i dopunjen nulama, kao na primer: \verb|pt2-0023.bmp|.

Isti \v sablon imena fajla se koristi i za snimanje i za u\v citavanje. Ne postoji komanda za poni\v stavanje
izmene plo\v cice sadr\v zajem fajla, tako da je koristite obazrivo. Za vi\v se detalja pogledajte
\nameref{replacingTiles}.

``Ubacite sve plo\v cice'' komanda \'ce pretra\v ziti trenutni direktorijum i ubaciti prvi neprekidni opseg
bitmapa imenovane prema prethodno pomenutom \v sablonu.

U slu\v caju da \v zelite da ubacite vi\v se plo\v cica nego \v sto teren trenutno sadr\v zi, mo\v zete upotrebiti
{\bf CONTROL}+{\bf INSERT} komandu koja \'ce dodati novu praznu plo\v cicu na kraj niza, i koja mo\v ze
poslu\v ziti kao odredi\v ste za ubacivanje iz fajla. Ovo je predvi\dj eno za ubacivanje manjeg broja plo\v cica,
za masovno ubacivanje koristite {\bf SHIFT}+{\bf INSERT} komandu.

Kada se ozna\v ci plo\v cica, ona \'ce tako\dj e biti ozna\v cena i u re\v zimu terena, i obratno. Prilikom izmene
broja plo\v cice, stara i nova jednostavno zamene mesta. Brisanje plo\v cica funkcioni\v se isto kao i u re\v zimu
terena i obrisana plo\v cica se mo\v ze povratiti.

\vspace{\baselineskip}

\noindent Komande:

\vspace{\baselineskip}

\begin{tabular}{lll}
{\bf STRELICE DESNO/LEVO} &--& naredni/prethodni teren\\
{\bf STRELICE GORE/DOLE}  &--& naredna/prethodna plo\v cica\\
{\bf HOME/END}            &--& prva/poslednja plo\v cica\\
{\bf PAGE UP/DOWN}        &--& 10 plo\v cica napred/nazad\\
{\bf SPACE/BACKSPACE}     &--& promena tipa (uslova) terena\\
{\bf F2}                  &--& snimi plo\v cicu u bitmapu\\
{\bf SHIFT}+{\bf F2}      &--& snimi sve plo\v cice u bitmape\\
{\bf G}                   &--& direktan skok na plo\v cicu\\
{\bf C}                   &--& izmena broja plo\v cice\\
{\bf H}                   &--& ozna\v ci trenutnu plo\v cicu\\
{\bf DELETE}              &--& obri\v si trenutnu plo\v cicu\\
{\bf INSERT}              &--& zameni trenutnu plo\v cicu sa\\
                          &  & odgovaraju\'com bitmapom\\
{\bf SHIFT}+{\bf INSERT}  &--& zameni sve plo\v cice u direktorijumu\\
{\bf CONTROL}+{\bf INSERT} &--& ubaci praznu plo\v cicu na kraj\\
{\bf CONTROL}+{\bf Z}     &--& poni\v sti poslednju operaciju\\
{\bf CONTROL}+{\bf Y}     &--& ponovi poslednju poni\v stenu operaciju\\
\end{tabular}

\begin{figure}[H]
  \begin{center}
    \includegraphics[width=320px,height=200px]{images/tile-mode}
    \caption{Re\v zim plo\v cica}
  \end{center}
\end{figure}

\subsubsection{Izmena plo\v cica}\label{replacingTiles}

Procedura zamene plo\v cica je sli\v cna proceduri zamene sprajtova, s tim \v sto su ovde jedine dozvoljene
dimenzije bitmape 16 x 16. Pikseli bitmape se prosto kopiraju na odgovaraju\'ce mesto u fajlu terena. Indeksni
fajl terena se mo\v ze izmeniti u re\v zimu terena.

Prilikom obrade plo\v cice u grafi\v ckom editoru, treba da znate da \'ce boje 10 i 11 biti prepisane. Boja 10
\'ce biti boja \v sorca ako tim koristi jednobojni dres, ili boja pozadine dresa u suprotnom. Boja 11 \'ce uvek
biti osnovna boja dresa. Ova promena boja se primenjuje samo na prve 42 plo\v cice, i to su uglavnom plo\v cice
koje \v cine publiku. Te boje su crvena i plava, i mo\v zete videti u re\v zimu terena da su navija\v ci uglavnom
njima prekriveni.

Tipovi terena se posti\v zu izmenom palete. Boje u paleti: 0, 7, 9, 78, 79, 80,
81, 106 i 107 se menjaju u zavisnosti od tipa terena i time se posti\v zu
vizualne promene u skladu sa uslovima na terenu.

\v Sabloni od 1 do 24 imaju posebna svojstva -- oni su animirana publika.
\v Sabloni 1..12 su gornja publika, a \v sabloni 13..24 su donja. Neparni
indeksi su stati\v cne slike, a parni indeksi animirane (navija\v ci koji
ska\v cu, ma\v su \v salovima itd.). \v Sablon 0 je uglavnom prazna plo\v cica
(izuzev u slu\v caju trening terena).

Zamena plo\v cica je poprili\v cno bezbedna operacija, dok se na \v citanje
celog terena iz bitmape mora obratiti posebna pa\v znja. Ukoliko se zamenjuje
trening teren, slede\'ce ne va\v zi, jer taj teren nema publiku, pa ni
animirane plo\v cice. Kada se menja teren sa publikom, nulta plo\v cica \'ce
biti prazna (popunjena nulama), a plo\v cice od 1 do 24 \'ce biti pode\v sene:
1 = 2, 3 = 4, itd., svaki paran broj i naredni neparan \'ce sadr\v zati istu
plo\v cicu. To \'ce prouzrokovati isklju\v cenje animacije. Da bi animacija
ponovo proradila, potrebno je izmeniti brojeve animiranih plo\v cica u neparne
brojeve u opsegu 1..24. Parne animirane plo\v cice moraju biti ru\v cno
unesene.

Bitna stvar kod uno\v senja terena iz bitmape je da maksimalni broj
jedinstvenih plo\v cica na terenu iznosi 296. SWOS koristi samo 75,776 bajtova
memorije za plo\v cice, i ako se ta granica prekora\v ci, svaka plo\v cica sa
indeksom vi\v sim od 295 bi\'ce prepisana.

\newpage
\subsection{Re\v zim repriza}

Ovaj re\v zim je u osnovi eksterni program za pregled SWOS-ovih *.rpl i *.hil
fajlova. U glavnom meniju birate izme\dj u higlights i replay fajlova. Ukoliko
takvih fajlova ima u trenutnom direktorijumu, pojavi\'ce se meni sa imenima
fajlova. Postavite kursor na \v zeljeni fajl i pritisnite enter. Uz malo
sre\'ce projekcija po\v cinje.

\vspace{\baselineskip}

Tasteri dostupni u menijima, izuzev strelica, su:

\vspace{\baselineskip}

\begin{tabular}{lll}
{\bf BACKSPACE} &--& povratak u prethodni meni (ako postoji)\\
{\bf HOME}      &--& skok na prvu stavku menija\\
{\bf END}       &--& skok na poslednju stavku menija
\end{tabular}

\begin{verbatim}

\end{verbatim}

Za vreme snimka, mo\v zete koristiti slede\'ce naredbe:

\vspace{\baselineskip}

\begin{tabular}{lll}
{\bf ESCAPE} &--& prekid snimka\\
{\bf F}      &--& uklju\v ci/isklju\v ci ispis broja frejmova u sekundi\\
{\bf +/-}    &--& ubrzaj/uspori\\
{\bf P}      &--& pauza\\
{\bf N}      &--& (za vreme pauze) prika\v zi slede\'ci frejm
\end{tabular}

\begin{verbatim}

\end{verbatim}

Imajte u vidu da podprogram za pregled nije kompletan. Jo\v s uvek nedostaju
neke stvari poput zvuka, pokretnih reklama, animirane publike i njihovih boja.

\begin{figure}[H]
  \begin{center}
    \includegraphics{images/replay-mode}
    \caption{Re\v zim repriza}
  \end{center}
\end{figure}


\newpage
\section{Poznati bagovi/ograni\v cenja}

Ovaj program je dat takav kakav je, nema podr\v ske niti garancije, i ne\'ce vi\v se biti novih zvani\v cnih
verzija (ali nikad ne reci nikad -- kao na primer upravo sada).

Neke od poslednjih izmena su brzinski izhakovane i mo\v ze se de\v savati da program povremeno pukne. Ukoliko je
to ba\v s intenzivno i jako vas uzrujava, mo\v zete poku\v sati da me kontaktirate. Verovatno ne\'cu popraviti,
ali ako uklju\v cite detaljne informacije sa snimcima ekrana, videom i ta\v cnim koracima za reprodukovanje baga,
\v cuda su mogu\'ca. Potra\v zite tako\dj e fajl po imenu \verb|errlog.txt| koji mo\v ze sad\v zati korisne
informacije.

Uz sve ovo, ipak se nadam da \'ce se program pokazati korisnim, jer znam da lo\v s dizajn i slab korisni\v cki
interfejs nikada nisu mogli da zaustave {\bf istinske SWOS manijake :)}

\vspace{\baselineskip}
\noindent Sve najbolje,\\
Z.

\end{document}
